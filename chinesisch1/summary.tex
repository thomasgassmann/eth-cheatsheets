\documentclass[UTF8]{ctexart}
\usepackage[UTF8]{ctex}

\begin{document}

\section{Ausprache}

\subsection{Ausnahmen}

\begin{itemize}
    \item Folgt nach dem Wort 不 bù eine Silbe in einem vierten Ton, dann wird bù nicht bù,
    sondern bú im zweiten Ton ausgesprochen. Und den Tonwechsel muss man
    schreiben. Z.B. bú shì
    \item wenn zwei Silben im dritten Ton nebeneinander stehen die erste Silbe im zweiten
    Ton gelesen ABER NICHT geschrieben wird
    \item 还是 kann auch hái shi gelesen werden.
\end{itemize}

\subsection{一}

一 ist normalerweise im ersten Ton. Wenn aber 一 in Kombination mit einer anderen Zahl verwnedet wird, ändert sich immer der Ton (und wird auch so geschrieben). 

Wenn die Silbe danach ein vierter Ton ist, wird daraus ein zweiter Ton (gleich wie oben). In allen anderen Fällen (bei einer Kombination) wird es im vierten Ton ausgesprochen.

\begin{itemize}
    \item 一 (yī)
    \item 一百 (yìbǎi)
    \item 一万 (yíwàn)
\end{itemize}

Mehr (https://yoyochinese.com/blog/mandarin-chinese-tone-change-rules-yi-one)

Wenn auf 一 ein ZEW folgt, wird es im vierten Ton ausgesprochen.

Wenn man eine Telefonnummer sagt, darf man nicht yī sagen, sondern muss yāo sagen (und schreiben 幺).

\subsection{两}

两 wird generell verwendet als eine Zähleinheit. 二百 wird nur verwendet falls exakt 200 gesagt werden soll. 两百 ist auch möglich und wird generell bevorzugt. 二百零一 ist nicht möglich, nur 两百零一 ist möglich.

\subsection{Zahlen}

Nullen hintendran darf man weglassen. Ansonsten immer mit 零 "auffüllen". Bei Zehnern darf man Zahl am Ende weglassen.

\begin{itemize}
    \item 101: 一百零一
    \item 120: 一百二(十)
\end{itemize}

Siehe auch Video auf OLAT (2022-10-17).

Man kann auch 一万万 anstelle von 一亿 verwenden.


\subsection{Schwacher Ton}

Siehe S.77 in Zhongguohua für richtige Aussprache und Höhe.

\section{Grammatik}

\subsection{Satzstruktur}

Immer Subjekt, Angabe, Prädikate, Objekt, Modalpartikel.
吗 ist beispielsweise ein Modalpartikel. 怎么 kommt immer vor dem Verb. Ein Objekt kann auch ein Objektsatz sein, welcher wiederrum nach derselben Struktur aufgebaut ist.

\subsection{Mehr}

Siehe Dokumente auf OLAT.

\subsection{给}

给 kann verwendet werden um ein Objekt in die Angabe zu schieben. Beispielsweise gibt es in 你给我发短信吧 zwei Objekte, deshalb können wir ein Objekt (我) mit 给 in die Angabe stellen. Die Angabe ist dann 给我. Siehe auch Video auf OLAT (2022-10-17).

\subsection{Richtungen}

Dasselbe für Norden, Süden, etc.:

\begin{itemize}
    \item Im Norden von: 北方
    \item nördlich von: 北边
\end{itemize}

Für Nordosten, Südosten, etc. schreibt man zuerst W/E und erst dann N/S. 方 kann weggelassen werden wenn schon 2 Silben vorhanden sind (bspw. 西藏在中国西南).

\section{Geographie}

Seite 70-72 im Buch Provinzen mit Hauptstädten und Aussprache. Karte auf Seite 48.

\end{document}
