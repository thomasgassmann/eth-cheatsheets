\section*{Gaussian Mixture Model}

Assume that data is generated from a convex-combination of Gaussian distributions: \\[-8pt]

$p(x  | \theta) = p(x  | \mu, \Sigma, w) = \sum_{j=1}^k w_j \mathcal{N}(x; \mu_j, \Sigma_j)$

We don't have labels and want to cluster this data. The problem is to estimate the param. for the Gaussian distributions.

\ \ $\text{argmin}_{\theta} \; - \sum_{i=1}^n \log \sum_{j=1}^k w_j \cdot \mathcal{N}(x_i \; | \; \mu_j, \Sigma_j)$

This is a non-convex objective. Similar to training a GBC without labels. Start with guess for our parameters, predict the unknown labels and then impute the missing data. Now we can get a closed form update.

\subsection*{Hard-EM Algorithm}

\textbf{E-Step}: predict the most likely class for each data point:
\begin{align*}
	z_i^{(t)} &= \argmax{z} \; p(z \; | \; x_i, \theta^{(t-1)}) \\[-5pt]
	&= \argmax{z} \; p(z \; | \; \theta^{(t-1)}) \cdot p(x_i \; | \; z, \theta^{(t-1)})
\end{align*}
\textbf{M-Step}: compute MLE of $\theta^{(t)}$ as for GBC. \smallskip

Problems: labels if the model is uncertain, tries to extract too much inf. Works poorly if clusters are overlapping. With uniform weights and spherical covariances is equivalent to k-Means with Lloyd's heuristics.

\subsection*{Soft-EM Algorithm}

\textbf{E-Step}: calculate the cluster membership weights for each point ($w_j = \pi_j = p(Z = j))$: \\[-8pt]

\quad $\gamma_j^{(t)}(x_i) = p(Z = j \; | \; D) =\frac{w_j \cdot p(x_i ; \theta_j^{(t-1)})}{\sum_k w_k \cdot p(x_i ; \theta_k^{(t-1)})}$
		
\textbf{M-Step}: compute MLE with closed form:

$w_j^{(t)} = \frac{1}{n} \sum_{i=1}^n \gamma_j^{(t)}(x_i) \quad \; \mu_j^{(t)} = \frac{\sum_{i=1}^n x_i \cdot \gamma_j^{(t)}(x_i)}{\sum_{i=1}^n \gamma_j^{(t)}(x_i)}$

\qquad \quad $\Sigma_j^{(t)} = \frac{\sum_{i=1}^n \gamma_j^{(t)}(x_i)(x_i - \mu_j^{(t)})(x_i - \mu_j^{(t)})^\top}{\sum_{i=1}^n \gamma_j^{(t)}(x_i)}$


Init. the weights as uniformly distributed, rand. or with k-Means++ and for variances use spherical init. or empirical covariance of the data. Select $k$ using cross-validation.

\subsection*{Degeneracy of GMMs}

GMMs can overfit with limited data. Avoid this by add $v^2 I$ to variance, so it does not collapse (equiv. to a Wishart prior on the covariance matrix). Choose $v$ by cross-validation.

\subsection*{Gaussian-Mixture Bayes Classifiers}

Assume that $p(x \; | \; y)$ for each class can be modelled by a GMM.

\qquad $p(x \; | \; y) = \sum_{j=1}^{k_y} w_j^{(y)} \mathcal{N}(x; \mu_j^{(y)}, \Sigma_j^{(y)})$

Giving highly complex decision boundaries:

\qquad $p(y \; | \; x) = \frac{1}{z} p(y)  \sum_{j=1}^{k_y} w_j^{(y)} \mathcal{N}(x; \mu_j^{(y)}, \Sigma_j^{(y)})$

\subsection*{GMMs for Density Estimation}

Can be used for anomaly detection or data imputation. Detect outliers, by comparing the estimated density against $\tau$. Allows to control the FP rate. Use ROC curve as evaluation criterion and optimize using CV to find $\tau$.

\subsection*{General EM Algorithm}

\textbf{E-Step}: Take the expected value over latent variables $z$ to generate likelihood function $Q$:
\begin{align*}
	Q(\theta ; \theta^{(t-1)}) &= \E_{Z}[ \log  p(X, Z \; | \; \theta) \; | \; X, \theta^{(t-1)}] \\[-5pt]
	&= \sum_{i=1}^n \sum_{z_i=1}^k \gamma_{z_i}(x_i) \log p(x_i, z_i \; | \; \theta)
\end{align*}
with $\gamma_z(x) = p(z \; | \; x, \theta^{(t-1)})$

\textbf{M-Step}: Compute MLE / Maximize:
$$\theta^{(t)} = \argmax{\theta} \; Q(\theta; \theta^{(t-1)})$$

We have monotonic convergence, each EM-iteration increases the data likelihood.
