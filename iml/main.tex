% Basic stuff
\documentclass[a4paper,11pt]{article}
\usepackage{scrextend}

% 3 column landscape layout with fewer margins
\usepackage[landscape, left=0.75cm, top=0.75cm, right=0.75cm, bottom=1cm, footskip=15pt]{geometry}
\usepackage{flowfram}
\usepackage{bbm}
\ffvadjustfalse
\setlength{\columnsep}{0.5cm}
\Ncolumn[<10]{4}
\onecolumn[10]

% define nice looking boxes
\usepackage[many]{tcolorbox}
\changefontsizes[11pt]{11pt}

% a base set, that is then customised
\tcbset {
	base/.style={
		boxrule=0mm,
		leftrule=1mm,
		left=1.75mm,
		arc=0mm, 
		fonttitle=\bfseries, 
		colbacktitle=black!10!white, 
		coltitle=black, 
		toptitle=0.75mm, 
		bottomtitle=0.25mm,
		title={#1}
	}
}

\definecolor{brandblue}{rgb}{0.34, 0.7, 1}
\newtcolorbox{mainbox}[1]{
	colframe=brandblue, 
	base={#1}
}

\newtcolorbox{subbox}[1]{
	colframe=black!20!white,
	base={#1}
}

% Mathematical typesetting & symbols
\usepackage{amsthm, mathtools, amssymb} 
\usepackage{marvosym, wasysym}
\allowdisplaybreaks

% Tables
\usepackage{tabularx, multirow}
\usepackage{makecell}
\usepackage{booktabs}
\renewcommand*{\arraystretch}{2}

% Make enumerations more compact
\usepackage{enumitem}

% To include sketches & PDFs
\usepackage{graphicx}

% For hyperlinks
\usepackage{hyperref}
\hypersetup{
	colorlinks=true
}
\renewcommand{\baselinestretch}{.9}
% Metadata
\title{Cheatsheet\\ Introduction to Machine Learning}
\author{Thomas Gassmann}
\date{\vspace{-10pt}Sommer 2024}

% Math helper stuff
\def\limxo{\lim_{x\to 0}}
\def\limxi{\lim_{x\to\infty}}
\def\limxn{\lim_{x\to-\infty}}
\def\R{\mathbb{R}}
\def\P{\mathbb{P}}
\def\F{\mathcal{F}}
\def\sumn{\sum_{n=0}^\infty}
\def\sumk{\sum_{k=1}^\infty}
\def\E{\mathbb{E}}
\DeclareMathOperator{\Var}{\text{Var}}
\newcommand{\middot}{~\textperiodcentered~}
\newlist{rowlist}{enumerate*}{1}
\setlist[rowlist]{label={\textbf{\roman*}\text{: }}, afterlabel={}, itemjoin=\middot}

\newcommand{\C}{\mathbb{C}}
\newcommand{\K}{\mathbb{K}}
\newcommand{\N}{\mathbb{N}}
\newcommand{\Q}{\mathbb{Q}}
\newcommand{\Z}{\mathbb{Z}}
\newcommand{\X}{\mathbb{X}}
\renewcommand{\P}{\mathbb{P}}

\newcommand{\No}{\mathcal{N}}

\newcommand{\br}{\par\medskip\noindent}
% \newcommand{\P}{\mathbb{P}}
\newcommand{\Oh}{\mathcal{O}}
% \newcommand{\vphi}{\varphi}
% \newcommand{\veps}{\varepsilon} 
\newcommand{\bd}{\textbf}
\newcommand{\equi}{\Leftrightarrow}
\newcommand{\imp}{\Rightarrow}
\newcommand{\emp}{\varnothing}
\newcommand{\subs}{\subseteq}
\newcommand{\ol}{\overline}
\newcommand{\ra}{\rangle}
\newcommand{\la}{\langle}
\newcommand{\ox}{\otimes}
\newcommand*\dx{\mathop{}\!\mathrm{d}x}
\newcommand*\dy{\mathop{}\!\mathrm{d}y}
\newcommand*\dz{\mathop{}\!\mathrm{d}z}

\DeclareMathOperator*{\argmax}{arg\,max}
\DeclareMathOperator*{\argmin}{arg\,min}
\DeclareMathOperator*{\vol}{vol}
\DeclareMathOperator*{\Mat}{Mat}

\setlength{\parindent}{0pt}

\makeatletter
\renewcommand{\section}{\@startsection{section}{1}{0mm}%
                                {0pt}%
                                {0.5pt}%x
                                {\sffamily\bfseries\large}}
\renewcommand{\subsection}{\@startsection{subsection}{1}{0mm}%
                                {0pt}%
                                {0.1pt}%x
                                {\sffamily\bfseries}}
\begin{document}
\small

\section*{Kernels}
\subsection*{Examples of Kernels}
$k(x,y) = (x^\top y)^m$  all monomials of deg. m \\
$k(x,y) = (1+x^\top y)^m$ all monomials up to deg. m\\
There are $\binom{d+m}{m}=O_d(d^m)=O_m(m^d)$ monomials of 
order $m$ in $d$ variables.\\


\subsection*{Properties $k(x,y) = \phi(x)^\top \phi(y)$}


\subsection*{Valid kernels}
$f(k)$, where $f$ is a polynomial/power series with non-negative coefficients;\\
$k(\binom{x}{y}, \binom{x'}{y'})=k(x,x')k(y,y'),\\ 
k(\binom{x}{y}, \binom{x'}{y'})=k(x,x') + k(y,y')$ \\
where $\binom{x}{y}$ is concatenation of vectors;\\
$k(x,y)=g(x)k(x,y)g(y)$ where $g\colon X\to\R$.
If for $g$ all Taylor coefficients non-negative, then $k(x, x') = g(\left< x, x' \right>)$ is a valid kernel

\subsection*{Kernelized Ridge}
Ansatz: $w^*=\Phi^\top\alpha$\\
$\min_w\|\Phi w-y\|^2 + \lambda ||w||_2^2\\
=\min_a ||K\alpha -y||_2^2 + \lambda \alpha^\top K \alpha$\\
$\alpha^*=(K+\lambda I)^{-1} y$\\
Prediction: $\hat{y} = \Sigma_{i=1}^n \alpha_i^* k(x_i,x)$\\
\section*{Probabilistic Modeling}
\subsection*{MLE}
Given a choice of marginal $P(Y|X,\theta)$ take 
$\theta^* =\argmax_\theta \prod_{i=1}^n {P_\theta}(y_i|x_i).$

\subsection*{Bayes optimality}
$\argmin_f\E_{x,y}[(y-f(x))^2]=\E[Y\mid X]$\\
$\argmin_f\E_{x,y}[1_{[y\neq f(x)]}]\\=\argmax_yp(Y=y\mid X=x)$

\subsection*{Bias-Variance-noise decomposition}
$\E_{x,y}[(\hat{f}_D(x)- y)^2]\\=
\E_x[\E_D[\hat{f}_D(x)]-f^*(x)]^2\\
\hspace*{0.1mm}+\E_x[\operatorname{var}[\hat{f}_D(x)]]\\
\hspace*{0.1mm}+\E_{x,y}[(y-f^*(x))^2]$ where $f^*=\E[Y\mid X]$.


\subsection*{Logistic regression}
Parametrize $P(y\mid x)$ by $\frac{1}{1+\exp(-y w^T x)}$.\\
MLE is $\operatorname{argmax_w} P(y_{1:n}|w,x_{1:n})\\
= \operatorname{argmin_w} - \sum_{i=1}^n \log P(y_i|w,x_i)\\
= \operatorname{argmin_w} \sum_{i=1}^n \log(1+\exp(-y_i w^T x_i))$

\subsection*{Gradient for logistic regression}
$\ell(w) = \log(1+\exp(-yw^Tx))$\\
$\nabla_w \ell(w) =\frac{-yx}{1+\exp(yw^Tx)}$

\subsection*{Multiclass Logistic Regression}
Parametrize $P(Y=i\mid x)$ by $\frac{\exp(w_i^Tx)}{\sum_j \exp(w_j^Tx)}$.

\subsection*{Kernelized logistic regression}
$\min_\alpha\sum_i\log(1+\exp(-y_i\alpha^\top K_i)) + \lambda\alpha^\top K \alpha$
$\hat{P}(y\mid x)=\frac{1}{1+\exp(-y\sum_i\alpha_ik(x_i,x))}$
\\
\section*{Generative Modeling}
Discriminative: Estimate $P(y\mid x)$\\
Generative: Estimate $P(y,x)$

Typical approach to generative modeling:
\begin{enumerate}[noitemsep,leftmargin=6mm,topsep=2pt,parsep=2pt,partopsep=2pt]
    \item Estimate prior on labels $P(y)$
    \item Estimate conditional distribution $P(x\mid y)$ for each class y
    \item Obtain predictive distribution using Bayes' rule:
$P(y\mid x) = \frac{P(y) P(x\mid y)}{P(x)} = \frac{P(x,y)}{P(x)}$
\end{enumerate}

\subsection*{Decision rule}
$\hat{y} = \operatorname{argmax_{y}} P(y\mid x)\\
\hspace*{2.1mm}= \operatorname{argmax_{y}} P(y) \prod_{i} P(x_i\mid y)\\
\hspace*{2.1mm}= \operatorname{argmax_{y}} \log P(y) + \sum_{i} \log P(x_i\mid y)$

\subsection*{QDA/Gaussian Bayes Classifier}
$P(Y=y) = p_y$ and $P(x\mid y) = \mathcal{N}({\mu}_y, {\Sigma}_y)$\\
$\hat{p}_y= \frac{\operatorname{Count(Y = y)}}{n}$\\
$\hat{\mu}_{y} = \frac{1}{\operatorname{Count}(Y=y)} \sum_{i:y_i=y} {x_i} $\\
$\hat{\Sigma}_{y} = \frac{1}{\operatorname{Count}(Y=y)} \sum_{i:y_i=y} (x_i - \hat{\mu}_{y})(x_i-\hat{\mu}_y)^\top $

For two classes $\hat{y} = \operatorname{sign}\Big(\log\frac{P(Y=1\mid x)}{P(Y=-1\mid x)}\Big) $ 
\\ where  
$    \log\frac{P(Y=1\mid x)}{P(Y=-1\mid x)} = \log \frac{\hat{p}}{1-\hat{p}} + \frac{1}{2}\log \frac{|\hat{\Sigma}_-|}{|\hat{\Sigma}_+|}\\
    + \frac{1}{2}(x - \hat{\mu}_-)^\top \hat{\Sigma}_-^{-1} (x - \hat{\mu}_-) - \frac{1}{2}(x - \hat{\mu}_+)^\top \hat{\Sigma}_+^{-1} (x - \hat{\mu}_+)$

\subsection*{Gaussian Naive Bayes}
GBC with diagonal $\Sigma$s. GNB with shared $\Sigma$s across
two classes yields the same predictions as Logistic Regression
(if model is true).

\subsection*{Fisher's LDA (Subcase of GBC)}
Assume: Two classes, $p = 0.5$, ${\Sigma}_- = {\Sigma}_+ $

\subsection*{Outlier Detection}
Classify $x$ as outlier if $P(x) \leq \tau$.

\subsection*{Regularization}
\begin{itemize}[noitemsep,leftmargin=6mm,topsep=0pt,parsep=0pt,partopsep=0pt]
    \item Restricting model (i.e. covariance)
    \item Prior on parameters.\\
\end{itemize}

\section*{Mixture Models}
\subsection*{Gaussian Mixtures}
$P(x\mid z) = \sum_iw_i\mathcal{N}({\mu}_i, {\Sigma}_i)$\\
MLE is a nonconvex problem $\rightarrow$ EM.

\subsection*{Hard-EM}
\textbf{E-step: } Compute \\$z_i^{(t)} = \operatorname{argmax_z} P(z\mid x_i, \theta^{(t-1)})\\
\hspace*{1.4mm}= \operatorname{argmax_z} P(z\mid \theta^{(t-1)}) P(x_i\mid z,\theta^{(t-1)})\\
 \stackrel{\text{GMM}}{=}\operatorname{argmax_z} w_z^{(t-1)}\mathcal{N}(x;\ \mu_z^{(t-1)},\Sigma_z^{(t-1)})$\\
\textbf{M-step: }
$\theta^{(t)} = \operatorname{argmax_\theta} P(x_{1:n},z^{(t)}_{1:n}\mid \theta)$\\
Hard-EM converges to a local maximum
of $P(x_{1:n},z^{(t)}_{1:n}\mid \theta)$. It tends to do poorly if clusters overlap. 
Hard-EM for GMM with $w_z=\frac{1}{k}, \Sigma_z=\sigma^2{I}$
is equivalent to k-means.

\subsection*{Soft-EM}
\textbf{E-step: \newline} Compute the distribution of $Z\mid x,\theta^{(t-1)}$, i.e. for each $x$ the responsibilities
\begin{equation*}
    \begin{aligned}
        P_{\theta^{(t-1)}}(Z=j\mid x) &= \frac{P(Z=j)P(x\mid Z=j)}{P(x)} \\
        &\stackrel{\text{GMM}}{=}
        \frac{w_j \mathcal{N}(x;\ \Sigma_j,\mu_j)}{\sum_k w_k \mathcal{N}(x;\ \Sigma_k,\mu_k)}.
    \end{aligned}
\end{equation*}

\textbf{M-step:} 
\begin{equation*}
    \begin{aligned}
        \theta^{(t)}&=\argmax_\theta 
        \mathbb{E}_{Z_{1:n}\mid x_{1:n},\theta^{(t-1)}}
        \big[\log P_\theta(x_{1:n},Z_{1:n})\big]\\
        &\stackrel{\text{iid\&cond.ind.}}{=}
        \sum_{i=1}^n \mathbb{E}_{Z_i\mid x_i,\theta^{(t-1)}}
        \big[\log P_\theta(x_i,Z_i)\big] \\
        &= \sum_{i=1}^n \sum_{j=1}^k P_{\theta^{(t-1)}}(Z_i=j\mid x_i)\log P_\theta(x_i,Z_i=j)
    \end{aligned}
\end{equation*}

GMM M-step:\\ $w_j^{(t)} \leftarrow \frac{1}{n} \sum_{i} \gamma_j^{(t)} (x_i)$; \\
$\mu_j^{(t)} \leftarrow \frac{\sum_{i} \gamma_j^{(t)} (x_i) x_i}{\sum_{i} \gamma_j^{(t)} (x_i)}\\
\Sigma_j^{(t)} \leftarrow \frac{\sum_{i} \gamma_j^{(t)}(x_i) (x_i - \mu_j^{(t)}) (x_i - \mu_j^{(t)})^\top}{\sum_{i} \gamma_j^{(t)}(x_i)} \{+\nu^2\mathbb{I}\}$\\ %\text{|}\gamma^{(t) = \gamma}$

The cluster size can be selected via CV.
EM converges to a local maximum for GMMs, dependent on
initialization.
\subsection*{Semi-Supervised Learning w/ GMMs:}
Set $P_{\theta^{(t-1)}}(Z=j\mid x) =1_{\{j = y\}}$ for labeled points
$(x,y).$
\\


\end{document}
